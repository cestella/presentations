\documentclass{beamer}
\usepackage{beamerthemesplit}
\usepackage{hyperref}
\usepackage{color}
\usepackage{listings}
\begin{document}
\title{Data Science : A Perspective}
\author{Casey Stella}
\date{February 25, 2014}

\frame{\titlepage} 

\frame{\frametitle{Table of Contents}\tableofcontents} 

\section{Introduction}

\frame{\frametitle{Hi, I'm Casey}
\begin{itemize}
\item Education
   \begin{itemize}
   \item B.S. in Computer Science from University of Louisiana at Monroe
   \item M.S. in Mathematics from Texas A\&M University with an emphasis in Theoretical Computer Science
   \end{itemize}\pause
\item I tend to work with ``Big Data''
   \begin{itemize}
   \item I am a consulting data scientist and big data architect at Hortonworks, an open source software company
   \item I was an architect/software engineer on the high performance indexing team at Explorys, a medical informatics startup based in the Cleveland Clinic
   \item I was a Research Geophysicist in the Oil Industry doing signal processing
   \item I built a VOIP network oriented toward WoW gamers
   \end{itemize}
\end{itemize}
}


\section{Data Science}

\subsection{The Skin}

\frame{\frametitle{Data Science : The Skin}
\begin{exampleblock}{}
  {\large ``A data scientist is a peculiar blend of developer and statistician that is capable of turning data into awesome.''}
  \vskip5mm
  \hspace*\fill{\small--- Josh Wills, Director of Data Science at Cloudera}
\end{exampleblock}
}

\frame{\frametitle{Data Science : The Skin}
\begin{exampleblock}{}
  {\large ``A data scientist is a statistician who lives in San Francisco.''}
  \vskip5mm
  \hspace*\fill{\small--- Snarky Internet Troll}
\end{exampleblock}
}

\frame{\frametitle{Data Science : The Skin }
\begin{exampleblock}{}
  {\large ``Data Science is an apple.''}
  \vskip5mm
  \hspace*\fill{\small--- Me} 
\end{exampleblock}
}

\subsection{The Flesh}

\frame{\frametitle{Data Science : The Flesh}
\begin{itemize}
\item Gathering and repurposing data that is otherwise lost or forgotten.\pause
\item Processing said data using
  \begin{itemize}
  \item Traditional statistics\pause
  \item Machine Learning\pause
  \item Natural Language Processing\pause
  \end{itemize}
\item Visualizing said data in a way which allows for an insight to be derived.
\end{itemize}
}

\subsection{The Core}

\frame{\frametitle{The Core : Data Gathering}
\begin{itemize}
\item Traditionally data analytics has been focused on limited, mostly numeric, data.\pause
\item Traditional statistical analysis focused on predicting some numerical quantity from curated, well planned data.\pause
\item Increased computing power and decreased cost left many data that could be gathered without a home in existing statistical models and therefore dropped.\pause
\item Data science is the expansion of traditional analytics to include data that is being dropped or is unstructured.
\end{itemize}
}

\frame{\frametitle{The Core : Unstructured Data}
\begin{itemize}
\item Some of the most interesting data gathered is not simple numerical or categorical data.
  \begin{itemize}
  \item For example, doctor's notes, radiologist reports, Facebook postings.\pause
  \end{itemize}
\item Natural language processing is a technique to make a computer begin to understand the written word.\pause
\item Sometimes the outputs are structured data and sometimes the outputs are insights themselves.
\end{itemize}
}

\frame{\frametitle{The Core : Machine Learning}
\begin{itemize}
\item A technique in computer science and statistics whereby a computer algorithm is presented data and learns patterns about the data.\pause
  \begin{itemize}
  \item e.g. One could use machine learning to predict whether a given tweet originated from a person based on their twitter history.\pause
  \end{itemize}
\item Traditional statistical models are generally well defined by a human and run over the data.\pause
\item Machine learning models have a human defining the input, but the machine develops an internal model based on examples of the data.\pause
\item Sometimes these approaches are at odds, but both techniques have merit and are used in the field.
\end{itemize}
}

\subsection{The Seeds}

\frame{\frametitle{The Seeds: The Skill Set}
\begin{itemize}
\item Statistics/Mathematics
\item Computer Science
\item Domain expertise
  \begin{itemize}
  \item Data science is applied to a domain, so domain expertise is a necessity.
  \end{itemize}
\end{itemize}
}

\subsection{The Ecology}

\frame{\frametitle{Who is eating the apple?}
\begin{itemize}
\item Computational power and storage has made keeping and analyzing massive amounts of data feasible.\pause
\item More and more industries are interested in leveraging this data to make decisions.
  \begin{itemize}
  \item Retail
  \item Healthcare
  \item Finance
  \item Oil and Gas
  \end{itemize}
\end{itemize}
}

\frame{\frametitle{An apple looking for a tree}
\begin{itemize}
\item Data science skill sets are necessarily cross-disciplinary.\pause
\item Universities are just starting to widen their programs to create cross-domain training for data science.\pause
\item Furthermore, data science as a practice is hard to commoditize/productize.\pause
\item This adds up to high demand and low supply.
\end{itemize}
}

\frame{\frametitle{Challenges}
\begin{itemize}
\item Not all analytics are possible
  \begin{itemize}
  \item This can be jarring to those consuming data science.\pause
  \end{itemize}
\item Extremely hyped discipline.\pause
\item Realistic methodologies to predict expected time to completion for data science tasks do not exist or are flawed deeply.\pause
\item Tooling is either extremely expensive (restricting range of analytics) or free and harder to use.\pause
\item Explaining insights gained can be extremely challenging.
\end{itemize}
}

\section{Conclusion}

\frame{\frametitle{Conclusion}
\begin{itemize}
\item Thanks for your attention
\item Questions?
\end{itemize}
}

\end{document}
