\documentclass{beamer}
\usepackage{beamerthemesplit}
\usepackage{multirow}
\usepackage{array}
\usepackage{hyperref}
\usepackage[T1]{fontenc}
\usepackage{inconsolata}
\usepackage{xcolor,colortbl}
\usepackage{natbib}
\usepackage{listings}
%%\newcommand{\newblock}{}
\DeclareGraphicsExtensions{.pdf,.png,.jpg}
\usetheme[pageofpages=of,% String used between the current page and the
                         % total page count.
          bullet=circle,% Use circles instead of squares for bullets.
          titleline=true,% Show a line below the frame title.
          alternativetitlepage=true,% Use the fancy title page.
          ]{Torino}
\definecolor{light-green}{RGB}{144,238,144}
\makeatletter
\setbeamertemplate{footline}
{
  \leavevmode%
  \hbox{%
  \begin{beamercolorbox}[wd=.333333\paperwidth,ht=2.25ex,dp=1ex,center]{author in head/foot}%
    \usebeamerfont{author in head/foot}\insertshortauthor~~\beamer@ifempty{\insertshortinstitute}{}{(\insertshortinstitute)}
  \end{beamercolorbox}%
  \begin{beamercolorbox}[wd=.333333\paperwidth,ht=2.25ex,dp=1ex,center]{title in head/foot}%
    \usebeamerfont{title in head/foot}\insertshorttitle
  \end{beamercolorbox}%
  \begin{beamercolorbox}[wd=.333333\paperwidth,ht=2.25ex,dp=1ex,right]{date in head/foot}%
    \usebeamerfont{date in head/foot}\insertshortdate{}\hspace*{2em}
%    \insertframenumber{} / \inserttotalframenumber\hspace*{2ex} % DELETED
  \end{beamercolorbox}}%
  \vskip0pt%
}
\makeatother
\begin{document}
\author{{\bf Casey Stella}}
\institute[Hortonworks]{\includegraphics[width=40px,height=17px]{logo}}
\title{{\bf NLP Structured Data Investigation on Non-Text}}
\date{April, 2015} 

\frame{\titlepage} 

\frame{\frametitle{Table of Contents}\tableofcontents} 

\section{Preliminaries}

\frame{\frametitle{Introduction}
\begin{itemize}
\item I'm a Principal Architect at Hortonworks
\item I work primarily doing Data Science in the Hadoop Ecosystem
\item Prior to this, I've spent my time and had a lot of fun
  \begin{itemize}
  \item Doing data mining on medical data at Explorys using the Hadoop ecosystem
  \item Doing signal processing on seismic data at Ion Geophysical using MapReduce
  \item Being a graduate student in the Math department at Texas A\&M in algorithmic complexity theory
  \end{itemize}
\end{itemize}
}

\frame{\frametitle{Domain Challenges in Data Science}
A data scientist has to merge analytical skills with domain expertise.
\begin{itemize}
  \item Often we're thrown into places where we have insufficient domain experience.
  \item Gaining this expertise can be challenging and time-consuming.
  \item Unsupervised machine learning techniques can be very useful to understand complex data relationships.
\end{itemize}\pause
\vfill
We'll use an unsupervised structure learning algorithm borrowed from NLP to look at medical data.
}

\section{Borrowing from NLP}

\frame{\frametitle{Word2Vec}
Word2Vec is a vectorization model created by Google ~\citep{DBLP:journals/corr/abs-1301-3781}
that attempts to learn relationships between words automatically given a large corpus of sentences.
\begin{itemize}
  \item Gives us a way to find similar words by finding near neighbors in the vector space with cosine similarity.\pause
  \item Uses a neural network to learn vector representations.\pause
  \item Recent work by Pennington, Socher, and Manning ~\citep{D14-1162} shows that the word2vec model is equivalent to weighting a word co-occurance matrix weighting based on window distance and lowering the dimension by matrix factorization. \pause
\end{itemize}
{\bf Takeaway:} The technique boils down, intuitively, to a riff on word co-occurence.  See here\footnote{http://radimrehurek.com/2014/12/making-sense-of-word2vec/} for more.
}

\frame{\frametitle{Clinical Data as Sentences}
Clinical encounters form a sort of sentence over time.  For a given encounter:
\begin{itemize}
\item Vitals are measured (e.g. height, weight, BMI).
\item Labs are performed and results are recorded (e.g. blood tests).
\item Procedures are performed.
\item Diagnoses are made (e.g. Diabetes).
\item Drugs are prescribed.
\end{itemize}
Each of these can be considered clinical ``words'' and the encounter forms a clinical ``sentence''.\\\pause
{\bf Idea:} We can use word2vec to investigate connections between these clinical concepts.
}

\section{Demo}
\frame{\frametitle{Demo}
As part of a Kaggle competition\footnote{https://www.kaggle.com/c/pf2012-diabetes}, 
Practice Fusion, a digital electronic medical records provider released depersonalized 
clinical records of 10,000 patients.  I ingested and preprocessed these records into 
197,340 clinical ``sentences'' using Pig and Hive.\pause
\vfill
MLLib from Spark now contains an implementation of word2vec, so let's use pyspark and IPython Notebook to explore this dataset on Hadoop.
}

\section{Questions}

\frame{\frametitle{Questions}
Thanks for your attention!  Questions? 
\begin{itemize}
\item Code \& scripts for this talk available on my github presentation page.\footnote{http://github.com/cestella/presentations/}
\item Find me at http://caseystella.com 
\item Twitter handle: @casey\_stella 
\item Email address: cstella@hortonworks.com
\end{itemize}
}

\frame{\frametitle{Bibliography}
\bibliography{NLP_on_non_textual_data}{}
\bibliographystyle{plain}
}

\end{document}
