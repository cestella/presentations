\documentclass[aspectratio=169
              ,usenames
              ,dvipsnames
              %uncomment to see notes
              %,handout
              ]{beamer}

\usetheme{metropolis}
%gets rid of bottom navigation bars
\definecolor{iicolor}{RGB}{232,150,70}
\setbeamercolor{frametitle}{bg=iicolor}
%\setbeamercolor{frametitle}{bg=white, fg=black}

\setbeamertemplate{footline}[frame number]{}

%gets rid of bottom navigation symbols
\setbeamertemplate{navigation symbols}{}

%gets rid of footer
%will override 'frame number' instruction above
%comment out to revert to previous/default definitions
\setbeamertemplate{footline}{}

%un-comment to see the notes
%\setbeameroption{show only notes} 

\usepackage{appendixnumberbeamer}
\usepackage{mathtools}
\usepackage{environ}
\usepackage{booktabs}
\usepackage{xcolor}
\usepackage{graphicx}
\usepackage{subcaption}
\captionsetup{compatibility=false}
\usepackage[scale=2]{ccicons}
\usepackage{hyperref}
\usepackage{pgfplots, pgfplotstable}
\usepackage{xparse}
\usepgfplotslibrary{dateplot}
\pgfplotsset{compat=1.12} % for better axis label placement

% Create a function for generating inverse normally distributed numbers using the Box–Muller transform
\pgfmathdeclarefunction{invgauss}{2}{%
  \pgfmathparse{sqrt(-2*ln(#1))*cos(deg(2*pi*#2))}%
}

\pgfmathsetseed{3}
% Initialise an empty table with a certain number of rows
\pgfplotstablenew[
    create on use/x/.style={create col/expr=\pgfplotstablerow},
    create on use/brown1/.style={
        create col/expr accum={
            (
                max(
                    min(
                        invgauss(rnd,rnd)*0.1+\pgfmathaccuma,
                        inf % Set upper limit here
                    ),
                    -inf % Set lower limit here
                )
            )
        }{0}
    },
    create on use/brown2/.style={
        create col/expr accum={
            (
                max(
                    min(
                        invgauss(rnd,rnd)*0.1+\pgfmathaccuma,
                        inf
                    ),
                    -inf
                )
            )
        }{0}
    },
    columns={x, brown1, brown2}]{201}\loadedtable 

\usetikzlibrary{decorations.text,graphs,graphs.standard,fadings,shapes.arrows,shadows}

\definecolor{mygray}{RGB}{208,208,208}
\definecolor{background}{RGB}{29,47,43}
\newcommand*{\mytextstyle}{\sffamily\small\bfseries\color{black!85}}
\newcommand{\arcarrow}[3]{%
   % inner radius, middle radius, outer radius, start angle,
   % end angle, tip protusion angle, options, text
   \pgfmathsetmacro{\rin}{2.3}
   \pgfmathsetmacro{\rmid}{2.8}
   \pgfmathsetmacro{\rout}{3.3}
   \pgfmathsetmacro{\astart}{#1}
   \pgfmathsetmacro{\aend}{#2}
   \pgfmathsetmacro{\atip}{5}
   \fill[mygray, very thick] (\astart+\atip:\rin)
                         arc (\astart+\atip:\aend:\rin)
      -- (\aend-\atip:\rmid)
      -- (\aend:\rout)   arc (\aend:\astart+\atip:\rout)
      -- (\astart:\rmid) -- cycle;
   \path[
      decoration = {
         text along path,
         text = {|\mytextstyle|#3},
         text align = {align = center},
         raise = -1.0ex
      },
      decorate
   ](\astart+\atip:\rmid) arc (\astart+\atip:\aend+\atip:\rmid);
}

\usepackage{xspace}
\newcommand{\themename}{\textbf{\textsc{metropolis}}\xspace}
\newcommand\Wider[2][3em]{%
\makebox[\linewidth][c]{%
  \begin{minipage}{\dimexpr\textwidth+#1\relax}
  \raggedright#2
  \end{minipage}%
  }%
}

\title{{\bf Software Engineering}}
\date{}
\subtitle{For Fun and Profit}

\begin{document}

\maketitle
\section{Introduction}
\begingroup
\Huge
\begin{frame}
\frametitle{Introduction}
\begin{center}
Hi, I'm Casey Stella!
\end{center}
\end{frame}
\endgroup

\frame{\frametitle{Introduction}
A little bit about the now...
\begin{itemize}
  \item I am a staff engineer at Stripe\pause
  \item Stripe is a fintech company which processes a good chunk of the internet's payments\pause
  \item Specifically, I'm the lead of the Machine Learning Infrastructure\pause
  \item Stripe uses a lot of machine learning: fraud detection, image recognition, etc.\pause
  \item We ensure ML models can run at the speed and scale of internet commerce.\pause
\end{itemize}
Open Questions:\\
  What even is software engineering?\\
  How in the heck did I end up doing this?
}

\section{Software Engineering}
\begingroup
\Huge
\begin{frame}
\frametitle{Programming: The Art}
\begin{center}
Programming is the art of turning
fuzzy ideas into working code.
\end{center}
\end{frame}
\endgroup

\begingroup
\Huge
\begin{frame}
\frametitle{Software Engineering: The Discipline}
\begin{center}
Software Engineering is the discipline of programming 
in such a way that your coworkers don't want to kill you
in your sleep\pause...often.
\end{center}
\end{frame}
\endgroup


\frame{\frametitle{Software Engineering: Unexpected Skills}
There are a few unexpected skills that you might not have thought to cultivate:\pause
\begin{itemize}
\item The Scientific Method: Software Engineering is an experimental science/engineering discipline.\pause
\item Logic and Reasoning: Most practical software is a complex set of systems with complex dependencies and expectations. Reasoning about structure is important.\pause
\item Soft Skills are more important than technical skills: Humans are more complex than software...by a lot.
\end{itemize}
}

\section{How in the heck did I get here?}

\frame{\frametitle{Education: Lessons Learned, Dues Paid}
BS in Computer Science / Math and a MS in Math, but what lessons were learned?\pause
\begin{itemize}
\item Liberal Arts Education: An investment in soft skills\pause
\item Don't Specialize too Early: Math taught me how to think in a structured way\pause
\item Choosing electives is like investing in yourself\pause
\item Go slow: Don't be in such a hurry to leave
\end{itemize}
}

\frame{\frametitle{Starting Up with a Startup}
First job out of grad school and I was out of the frying-pan and into the fire\pause
\begin{itemize}
\item Negotiating: I was a well-educated moron\pause
\item Academic programming $\ne$ Software Engineering: Interesting $\not \Rightarrow$ Lucrative\pause
\item Software floats on a sea of money: Learn about economics and business\pause
\item Startups are a mood, especially when they stop being startups\pause
\item Leaving is a Strategy
\end{itemize}
}

\frame{\frametitle{Oil Industry Days}
More Data $\implies$ More Problems, but some lessons\pause
\begin{itemize}
\item Careers are an optimization problem: Specialize to the area which maximizes your interests, difficulty of entry, and usefulness in making revenue\pause
\item Rule of Thumb: Be close enough to the revenue stream to get your toes wet but not drown\pause
\item Legacy codebases are like archaeological digs filled with dynamite: make changes with great care
\end{itemize}
}

\begingroup
\Huge
\begin{frame}
\frametitle{Rinse and Repeat}
\begin{center}
Wash, Rinse and Repeat: Sisyphus has nothing on me
\end{center}
\end{frame}
\endgroup

\frame{\frametitle{Consulting: That Billable Hours Grind}
Consulting in a hot Silicon Valley startup is still...consulting\pause
\begin{itemize}
\item Consulting: sometimes your job is to be the neck to choke\pause
\item Everything is negotiable\pause
\item Pigeonholes are just suggestions\pause
\item Sometimes you can do everything right and still fail
\end{itemize}
}

\frame{\frametitle{Stripe: The Present}
And now you're caught up...\pause
\begin{itemize}
\item Go where you're needed most (and where they're willing to pay)\pause
\item Different companies have different professional cultures\pause
\item Ambition can only take you so far: Knowing when to stop...
\end{itemize}
}

\frame{\frametitle{Career Guidance: A recap}
Things I wish I had known when I started:
\begin{itemize}
\item Learn how to negotiate\pause
\item Demonstrated curiosity and interest is the most desireable quality in any software engineer\pause
\item Soft Skills become more of the job as your career advances\pause
\item Develop thick skin against firing and judgement\pause
\item Learn how to grow within a company\pause
\item Know when to leave: Moving every 2 years early on is pretty good strategy for many\pause
\item Know what you like and what you don't.  Do the former, don't do the latter.\pause
\item Learn how to interview (it changes over time)
\end{itemize}
}

\frame{\frametitle{Questions}
Thanks for your attention!  Questions? 
}

\end{document}
